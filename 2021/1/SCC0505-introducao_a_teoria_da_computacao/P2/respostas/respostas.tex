\documentclass[12pt]{article}
\usepackage{amsmath}
\usepackage{amssymb}
\usepackage{commath}
\usepackage{systeme}
\usepackage{booktabs}
\usepackage{indentfirst}
\usepackage{mathtools}
\usepackage{hyperref}
\usepackage{bm}

\usepackage[portuguese]{babel}
\renewcommand{\thesubsection}{\thesection.\alph{subsection}}

\usepackage[small]{titlesec}
\titleformat{\part}[display]
  {\normalfont\large\bfseries}{\partname\ \thepart}{14pt}{\Large}

\title{SCC0505 \\ Introdução à Teoria da Computação \\ Prova 2}
\author{Gabriel Silva Fontes | 10856803}

\begin{document}
\maketitle

\section{}
\subsection{}
\emph{Incluído no arquivo Q1P2.jff}
\subsection{}
\(-\), \(\bm{q_0}\), \(0112\)

\(0\), \(\bm{q_1}\), \(112\)

\(01\), \(\bm{q_2}\), \(12\)

\(011\), \(\bm{q_2}\), \(2\)

\(0112\), \(\bm{q_3}\), \(-\)

\(0112\), \(\bm{q_a}\), \(-\)

\section{}
\subsection{}
\(S \rightarrow aSa | aBa\)

\(B \rightarrow bB | bC\)

\(C \rightarrow cC | c \)
\subsection{}
\emph{Incluído no arquivo Q2P2.jff}

\section{}
\emph{Incluído no arquivo Q3P2.jff}

\section{}
A linguagem \(L_d\) é indecidível, igual o problema da parada, pois esta não é recursivamente enumerável. Isso significa que não existe máquina de Turing capaz de decidir onde parar ao processar essa linguagem. Visto que essas apenas processam linguagens recursivamente enumeráveis, do tipo 0.

\end{document}
