\documentclass[12pt]{article}
\usepackage{amsmath}
\usepackage{amssymb}
\usepackage{commath}
\usepackage{systeme}
\usepackage{booktabs}
\usepackage{indentfirst}
\usepackage{mathtools}
\usepackage{minted}
\usepackage{hyperref}


\usepackage[portuguese]{babel}
\DeclarePairedDelimiter{\ceil}{\lceil}{\rceil}

\usepackage[small]{titlesec}
\titleformat{\part}[display]
  {\normalfont\large\bfseries}{\partname\ \thepart}{14pt}{\Large}

\title{SCC0505 \\ Introdução à Teoria da Computação \\ Trabalho 1}
\author{Gabriel Silva Fontes | 10856803
        \and
        Amanda Lindoso Figueiredo | 10784306
        \and
        Rafael Doering Soares | 10408410
        \and
        Felipe Moreira Neves de Souza | 10734651}

\begin{document}
\maketitle
\section{Sobre a solução}
A solução é robusta e funciona com DFAs e NFAs. Suportando um número de estados, símbolos, e transições praticamente infinito (dentro dos limites da memória do computador).

Fomos um passo além, e demos suporte também a cadeias e transições vazias (\(\epsilon\)). Suportando, assim, um NFA-\(\epsilon\).

É possível refatorar o programa para que o autômato aceite qualquer tipo de dado genérico como estado e símbolo. Podendo assim, talvez, ser utilizado para resolver problemas computacionais interessantes.

\section{O código}
\subsection{Linguagem e paradigma}
O código foi construído com a linguagem Rust, principalmente com o paradigma de Programação Orientada a Objetos, e algumas funcionalidades comumente presente em linguagens funcionais (iteradores, closures, etc).

\subsection{Estrutura Automaton}
No arquivo src/automaton.rs, temos a estrutura e métodos do autômato.
A definição da estrutura é:

\begin{minted}{rust}
pub struct Automaton {
    graph: Graph<(Option<u16>, bool), Option<char>>,
    initial_state: NodeIndex,
}
\end{minted}

Ela é formada por um grafo, cujos nós são uma tupla Option\footnote{No Rust, Option representa um tipo nulável. O valor do nó é nulável para dar suporte ao estado inicial, que não tem um valor. Também usamos nas arestas, onde o nulo representa uma aresta \(\epsilon\)} u16\footnote{u16 significa unsigned integer com 16 bits} e boolean. O u16 representa o estado, e a boolean marca se ele é ou não um estado final aceitável. As arestas são compostas por Option char, representando um símbolo na linguagem.

Também temos o índice do grafo onde está localizado o estado inicial.

\subsection{Construtor}
Essa classe possui um construtor (método new), cuja assinatura é:

\begin{minted}{rust}
pub fn new(
    states: &[u16],
    initial_states: &[u16],
    accepted_states: &[u16],
    transitions: &[(u16, char, u16)],
) -> Result<Automaton>
\end{minted}

Esse método toma um vetor de estados, um vetor de estados iniciais, um vetor de estados aceitáveis, e um vetor de transições. Retornando um Automaton construído.

O construtor se encarrega de alocar e inicializar o grafo, adicionar o nó inicial (os nós ``iniciais'' passados pelo usuário serão ligados a este por \(\epsilon\)), adicionar os nós de estados, e relacionar eles com as arestas (transições).

\subsection{Método de verificação}

Nossa classe oferece o componente de simulação como um método. Sua parte pública tem a seguinte assinatura:
\begin{minted}{rust}
pub fn verify_chain(&self, chain: &[char]) -> bool
\end{minted}

Esse método se encarrega de ser o caller inicial de um outro método (recursivo, por isso separado), com o nó (estado) inicial.

A função recursiva interna tem a seguinte assinatura:
\begin{minted}{rust}
fn verify_chain_inner(
    &self,
    current &NodeIndex,
    visited: Vec<&NodeIndex>,
    chain: &[char]
) -> bool
\end{minted}

Tomando um nó inicial, um vetor listando os nós que foram deixados por uma transição \(\epsilon\) (para evitar que uma transição \(\epsilon\) faça um loop infinito) e a cadeia a ser verificada, essa função trabalha de maneira recursiva para buscar qualquer caminho que lhe proporcionaria alcançar um nó aceito ao finalizar a cadeia.

Primeiro, é feita uma verificação se a cadeia chegou ao fim \textbf{e} o estado atual é aceitável, nesse caso retornamos true, sinalizando que a cadeia foi aceita.

Caso isso não aconteça, buscamos as arestas que correspondem ao símbolo atual, chamando a função novamente nos estados que vêm depois de todas essas (e propagando a aceitação, caso ocorra).

Depois desses casos (ou seja, menor preferência), filtramos apenas as arestas \(\epsilon\) saindo do nosso nó atual, e então chamamos a função nos estados que seguem estas transições (novamente, propagando a aceitação, caso ocorra). Nesse caso, passamos para a função o vetor de visitas, adicionando o nó que estamos deixando (para evitar retornar à ele e causar uma recursão infinita e stack overflow, como mencionado).

No fim de tudo isso (ou seja, não temos mais \(\epsilon\) ou símbolo na cadeia para seguir, e não estamos num nó aceito), retornamos false, sinalizando rejeição daquela tentativa.


Pela natureza recursiva, esse método conseguirá verificar de forma exaustiva aquela cadeia. Incluindo possibilidades como passar por um \(\epsilon\), ir e voltar de um nó, etc.

\subsection{Arquivo main}
Por uma questão de organização e encapsulamento, a lógica que se refere à entrada do usuário é contida no arquivo src/main.rs.

Esse código tem acesso à estrutura e métodos de Automaton, e se encarrega de ler, analisar, e chamar os métodos construtores e de verificação da estrutura Automaton.

Aqui temos uma estrutura representando a entrada, e um método para a receber da entrada padrão. Após sua leitura, iteramos pelas cadeias, chamando o método verify\_chain em cada uma, e imprimindo na tela se são aceitas ou não:

\begin{minted}{rust}
fn main() -> Result<()> {
    // Inicializar stdin
    let stdin = stdin();
    // Travar stdin e ler input
    // (estados, estados iniciais, estados aceitos, transições, e cadeias)
    let input = Input::from_reader(&mut stdin.lock())?;

    // Criar automato com os dados recebidos
    let automaton = Automaton::new(
        &input.states,
        &input.initial_states,
        &input.accepted_states,
        &input.transitions,
    )?;

    // Para cada cadeia
    for chain in input.chains {
        // Verificar se aceita ou rejeita
        if automaton.verify_chain(&chain) {
            println!("aceita")
        } else {
            println!("rejeita")
        }
    }

    Ok(())
}
\end{minted}
\section{Qualidade e eficiência}
Como mencionado anteriormente, é trivial alterar o programa para se obter uma implementação com Generics, permitindo o uso de qualquer tipo primitivo ou estrutura de dados no lugar dos estados e dos símbolos, sendo possível reutilizar o programa para  problemas potencialmente interessantes.

Sobre a linguagem, Rust emprega um verificador de memória conhecido como \textit{borrow checker}. Esta funcionalidade garante destrução da memória sem um garbage collector, e ótimas verificações (null pointer, dangling pointers, etc) já na compilação. Temos assim um código relativamente enxuto e intuitivo, com a mesma performance de linguagens com menos verificações de segurança.

A utilização de um grafo com acesso aleatório (em tempo \(O(1)\)) e tamanho pré-determinado permite uma grande eficiência ao inserir, acessar, buscar e filtrar arestas e nós da estrutura.

Na criação do autômato, percorremos uma única vez os estados e as arestas, sendo assim, esta operação tem complexidade (sendo \(e\) o número de estados e \(t\) o número de transições) \(O(e+t)\).

O método de verificação é um pouco mais complicado. Pela natureza recursiva, o pior caso, pode potencialmente visitar cada nó, para cada nó. Sendo assim, assume-se \(O(e^2)\).

A complexidade de espaço é a mínima possível. O armazenamento mais significativo na memória é o grafo, que contém os dados dos nós e arestas, apenas. Logo \(O(e+t)\).

\end{document}
