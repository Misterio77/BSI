\documentclass[12pt]{article}
\usepackage{amsmath}
\usepackage{amssymb}
\usepackage{commath}
\usepackage{systeme}
\usepackage{booktabs}
\usepackage{indentfirst}
\usepackage{mathtools}
\usepackage{minted}
\usepackage{hyperref}


\usepackage[portuguese]{babel}
\DeclarePairedDelimiter{\ceil}{\lceil}{\rceil}

\usepackage[small]{titlesec}
\titleformat{\part}[display]
  {\normalfont\large\bfseries}{\partname\ \thepart}{14pt}{\Large}

\title{SCC0505 \\ Introdução à Teoria da Computação \\ Trabalho 2}
\author{Gabriel Silva Fontes | 10856803
        \and
        Amanda Lindoso Figueiredo | 10784306
        \and
        Rafael Doering Soares | 10408410
        \and
        Felipe Moreira Neves de Souza | 10734651}

\begin{document}
\maketitle
\section{Sobre a solução}
A nossa implementação do simulador de máquinas de turing suporta fielmente todas as especificações nescessárias, e um pouco mais.

Mais especificamente: não temos limite algum na quantidade de símbolos, transições, ou estados; podemos ter mais que um estado final, o símbolo branco e o estado inicial também são configuráveis.

Assim como no T1, é possível generalizar e utilizar Generics para permitir o uso de qualquer tipo de dado como símbolo e estado.

\section{Código}
\subsection{Linguagem e paradigmas}
Assim como no anterior, desenvolvemos com Rust. Principalmente com Orientação a Objetos e algumas estratégias funcionais (em especial na leitura de input).


\subsection{Tipos usados}
É bem mais fácil expressar as transições num autômato determinístico: um par estado+símbolo vai sempre resultar em uma, e apenas uma, tripla estado, símbolo, direção. Por isso, utilizamos uma estrutura Map (que é um hashmap) para representá-las.

Os conjuntos são representados por uma estrutura Set (que é um hashset).

Os símbolos e estados podem ser editados para outros tipos de dado (ou generalizado para funcionar com todos), mas na nossa implementação estão definidos, respectivamente, como um caractere e um u16\footnote{Inteiro não-negativo de 16 bits}. Além disso, Direction é simplesmente um enum com 3 possibilidades (esquerda, direita, não mover).

\begin{minted}{rust}
type Symbol = char;
type State = u16;

enum Direction {
    Left,
    Right,
    None,
}
\end{minted}

\subsection{Estrutura TuringMachine}

Considerando os tipos explicados acima, essa é a definição da máquina, propositalmente muito similar a definição matemática:

\begin{minted}{rust}
pub struct TuringMachine {
    input_symbols: Set<Symbol>,
    tape_symbols: Set<Symbol>,
    blank_symbol: Symbol,
    states: Set<State>,
    initial_state: State,
    accepting_states: Set<State>,
    transitions: Map<(State, Symbol), (State, Symbol, Direction)>,
}
\end{minted}

\subsection{Método run\_tape}

O método que executa a fita tem a seguinte assinatura:
\begin{minted}{rust}
pub fn run_tape(&self, tape: Vec<Symbol>) -> bool
\end{minted}

Ou seja, em uma turing machine, recebendo uma fita (vetor de símbolos), retornamos se é ou não aceito (boleano).

A função basicamente executa um loop, que em cada iteração verifica qual a próxima transição partindo do nosso estado e símbolo atual.

Caso a transição esteja definida, escrevemos o símbolo, trocamos o estado, e andamos com a cabeça de leitura, expandindo a fita conforme nescessário.

Caso não esteja definida, significa que a máquina alcançou estado de halt. Nesse momento, vemos se o estado atual está ou não contido no conjunto de estados finais aceitos, e retornamos isso.

\subsection{Main}

Assim como no trabalho anterior, a main cuida de ler inputs, tratar esses dados, e os repassar para a TuringMachine, construindo a estrutura.

Feito isso, chamamos run\_tape para cada fita inserida no input, imprimindo aceita ou rejeita de acordo com o retorno da função.

\subsection{Qualidade e eficiência}
A implementação é muito intuitiva e elegante, graças ao determinismo. Acontece simplesmente uma iteração por ação da máquina.

Muito flexível e fácil de expandir, é trivial adicionar funcionalidades interessantes, como imprimir a fita final após o halt.
\end{document}
