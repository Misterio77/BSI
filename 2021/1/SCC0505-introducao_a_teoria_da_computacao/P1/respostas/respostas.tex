\documentclass[12pt]{article}
\usepackage{amsmath}
\usepackage{amssymb}
\usepackage{commath}
\usepackage{systeme}
\usepackage{booktabs}
\usepackage{indentfirst}
\usepackage{mathtools}
\usepackage{hyperref}

\usepackage[portuguese]{babel}
\renewcommand{\thesubsection}{\thesection.\alph{subsection}}

\usepackage[small]{titlesec}
\titleformat{\part}[display]
  {\normalfont\large\bfseries}{\partname\ \thepart}{14pt}{\Large}

\title{SCC0505 \\ Introdução à Teoria da Computação \\ Prova 1}
\author{Gabriel Silva Fontes | 10856803}

\begin{document}
\maketitle

\section{}
\subsection{}
\[
    S \rightarrow 0S1|011
\]

\subsection{}
É possível com o autômato de pilha (APN), está no arquivo 1b.jff.

\section{}
\subsection{}
É possível, está no arquivo 2a.jff.
\subsection{}
\[(1+2+3) *+321+(1+2+3) *\]

\section{}
\subsection{}
Está no arquivo 3a.jff.
\subsection{}
\begin{align*}
    S &\rightarrow aA | bC | \lambda \\
    A &\rightarrow aS | bB \\
    B &\rightarrow bA | aC \\
    C &\rightarrow aB | bS
\end{align*}
\subsection{}
Está no arquivo 3c.jff.

É não determinístico. Ao processar o símbolo \(a\) ou \(b\) com um \(Z\) na pilha, além das transições \(a,Z,A\) e \(b,Z,B\), a transição \(\lambda,Z,\lambda\) também é possível.

\end{document}
