\documentclass[12pt]{article}
\usepackage{amsmath}
\usepackage{amssymb}
\usepackage{commath}
\usepackage{indentfirst}
\usepackage{mathtools}
\usepackage{minted}
\usepackage{graphicx}

\usepackage[portuguese]{babel}
\DeclarePairedDelimiter{\ceil}{\lceil}{\rceil}

\usepackage[small]{titlesec}
\titleformat{\part}[display]{\normalfont\large\bfseries}{\partname\ \thepart}{14pt}{\Large}

\title{SME0500 \\ Cálculo Numérico \\ Atividade 7}
\author{Gabriel Silva Fontes | 10856803}
\begin{document}
\maketitle
\section{Os arquivos incluidos}
Além desse relatório, incluí os arquivos:

\begin{itemize}
    \item \(build\_system.m\): arquivo-função que gera o sistema, partindo das bases e do dataset;

    \item \(solve.m\): arquivo-script que lê dataset, define bases, utiliza a função do arquivo \(build\_system.m\) e resolve o sistema;

    \item \(graph.m\): arquivo-script que utiliza a solução de \(solve.m\) e o dataset para montar gráfico. Ele tem uma função que utiliza os coeficientes e bases para calcular o polinômio, mapeando valores de X e Y para seus respectivos Z.
\end{itemize}

\section{Código que constrói e resolve o sistema}
\subsection{Sobre}
Implementei a função no arquivo \(build\_system.m\). Ela toma um conjunto de handles de funções, representando as bases, e também uma matriz Nx3 (colunas com x, y, z), que representa o dataset de pontos.

As bases são definidas assim para adicionar grande flexibilidade na resolução, enquanto o dataset pode ser lido do input, escrito manualmente, ou facilmente carregado com \(importdata\).

Exemplifiquei como usar nas seções seguintes.

\subsection{Como funciona}
O código é relativamente simples.

Primeiro, montamos uma matriz de bases computadas. Isto é, para cada base, temos uma coluna representando os valores que ela retorna quando cada \(x\) e \(y\) do dataset é plugado nela.

Segundo, iteramos novamente nas bases, e começamos preenchendo \(b\), com os produtos escalares entre cada uma das bases previamente calculadas pelo conjunto de coordenadas \(z\), contendo um outro loop pelas bases, para construir a matriz \(A\) utilizando a mesma ideia de tirar os produtos escalares das bases previamente calculadas.

\section{Exemplo simples (Caso\_teste.txt)}
O primeiro caso foi ajustar aos pontos um polinômio da forma:
\[
    P_{22}(x,y) = \alpha_{00} + \alpha_{10}x + \alpha_{01}y + \alpha_{20}x^2 + \alpha_{11}xy + \alpha_{02}y^2
\]

\subsection{Código}
\begin{minted}{matlab}
data = importdata('Caso_teste.txt', '\t', 1).data;
bases = {
    @(x,y)(1),
    @(x,y)(x),
    @(x,y)(y),
    @(x,y)(x^2),
    @(x,y)(x*y),
    @(x,y)(y^2),
};

[A, b] = build_system(bases, data);
x = A\b;
\end{minted}

\subsection{Resultado}
\[
    x =
    \begin{bmatrix}
        0 \\
        -3 \\
        4 \\
        1 \\
        2 \\
        -1
    \end{bmatrix}
\]
\begin{center}
\includegraphics[scale=0.9]{grafico1.png}
\end{center}

\section{Exemplo avançado (Ex07\_Points.txt)}
O segundo caso foi ajustar ao outro arquivo de pontos um polinômio de grau maior:
\[
    P_{23}(x,y) = \alpha_{00} + \alpha_{10}x + \alpha_{01}y + \alpha_{20}x^2 + \alpha_{11}xy + \alpha_{02}y^2 + \alpha_{12}xy^2 + \alpha_{03}y^3
\]

\subsection{Código}
\begin{minted}{matlab}
data = importdata('Ex07_Points.txt', '\t', 1).data;
bases = {
    @(x,y)(1),
    @(x,y)(x),
    @(x,y)(y),
    @(x,y)(x^2),
    @(x,y)(x*y),
    @(x,y)(y^2),
    @(x,y)(x*y^2),
    @(x,y)(y^3),
};

[A, b] = build_system(bases, data);
x = A\b;
\end{minted}

\subsection{Resultado}
\[
    x =
    \begin{bmatrix}
        1.9103 \\
        -2.4358 \\
        -0.0988 \\
        0.6682 \\
        0.1930 \\
        -0.3648 \\
        0.0551 \\
        0.1082
    \end{bmatrix}
\]
\begin{center}
\includegraphics[scale=0.9]{grafico2.png}
\end{center}


\end{document}
