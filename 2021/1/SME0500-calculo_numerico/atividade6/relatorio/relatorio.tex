\documentclass[12pt]{article}
\usepackage{amsmath}
\usepackage{amssymb}
\usepackage{commath}
\usepackage{systeme}
\usepackage{xstring}
\usepackage{booktabs}
\usepackage{indentfirst}
\usepackage{mathtools}
\usepackage{tikz}
\usepackage{pgfplots}
\usepackage{float}
\pgfplotsset{compat=newest}

\usepackage[portuguese]{babel}
\DeclarePairedDelimiter{\ceil}{\lceil}{\rceil}

\usepackage[small]{titlesec}
\titleformat{\part}[display]
{\normalfont\large\bfseries}{\partname\ \thepart}{14pt}{\Large}

\title{SME0500 \\ Cálculo Numérico \\ Atividade 6}
\author{Gabriel Silva Fontes | 10856803}
\begin{document}
\maketitle

% \section*{2.}
% \subsection*{(a) \(P_1\)}
% \subsection*{(b) \(P_2\)}

\section*{4.}

\subsection*{(a)}

\subsubsection*{1. \(g_1(x) = ax^2+bx\)}

Na forma matricial:
\[
\begin{bmatrix}
    y_1 \\
    y_2 \\
    y_3 \\
    y_4
\end{bmatrix}
=
\begin{bmatrix}
    {x_1}^2 & x_1 \\
    {x_2}^2 & x_2 \\
    {x_3}^2 & x_3 \\
    {x_4}^2 & x_4 \\
\end{bmatrix}
\cdot
\begin{bmatrix}
    a \\
    b
\end{bmatrix}
\]
Colocando nossos valores:
\[
\begin{bmatrix}
    1 \\
    -3 \\
    1 \\
    9
\end{bmatrix}
=
\begin{bmatrix}
    4 & -2 \\
    1 & -1 \\
    1 & 1 \\
    4 & 2 \\
\end{bmatrix}
\cdot
\begin{bmatrix}
    a \\
    b
\end{bmatrix}
\]

Que nos resulta em:
\[
\begin{bmatrix}
    a \\
    b
\end{bmatrix}
=
\begin{bmatrix}
    1.1176 \\
    2
\end{bmatrix}
\]
Ou seja, teremos a função:
\[
    \boxed{g_1(x) = 1.1176x^2 + 2x}
\]

\subsubsection*{2. \(g_2(x) = cx^2+d\)}
Na forma matricial:
\[
\begin{bmatrix}
    y_1 \\
    y_2 \\
    y_3 \\
    y_4
\end{bmatrix}
=
\begin{bmatrix}
    {x_1}^2 & 1 \\
    {x_2}^2 & 1 \\
    {x_3}^2 & 1 \\
    {x_4}^2 & 1 \\
\end{bmatrix}
\cdot
\begin{bmatrix}
    c \\
    d
\end{bmatrix}
\]
Colocando nossos valores:
\[
\begin{bmatrix}
    1 \\
    -3 \\
    1 \\
    9
\end{bmatrix}
=
\begin{bmatrix}
    4 & 1 \\
    1 & 1 \\
    1 & 1 \\
    4 & 1 \\
\end{bmatrix}
\cdot
\begin{bmatrix}
    c \\
    d
\end{bmatrix}
\]

Que nos resulta em:
\[
\begin{bmatrix}
    c \\
    d
\end{bmatrix}
=
\begin{bmatrix}
    2 \\
    -3
\end{bmatrix}
\]
Ou seja, teremos a função:
\[
    \boxed{g_2(x) = 2x^2 - 3}
\]

\subsection*{(b)}

Vamos calcular os erros por truncamento. Primeiro do \(g_1\):
\begin{align*}
    Q_1 = {||f - g_1||}^2
    &= \sum^4_{i=1}{{(y_1 - g_1(x_i))}^2} \\
    &= \sum_{i=1}^{4}{y_i^2} - 2 \sum_{i=1}^{4}{y_i\cdot g_1(x_i)} + \sum_{i=1}^{4}{g_1^2(x_i)} \\
    &= 92 - 2\cdot(82.469) + 83.106 = 10.168
\end{align*}

Agora o \(g_2\):
\begin{align*}
    Q_2 = {||f - g_2||}^2
    &= \sum^4_{i=1}{{(y_1 - g_2(x_i))}^2} \\
    &= \sum_{i=1}^{4}{y_i^2} - 2 \sum_{i=1}^{4}{y_i\cdot g_2(x_i)} + \sum_{i=1}^{4}{g_2^2(x_i)} \\
    &= 92 - 2\cdot(52) + 28.380 = 16.380
\end{align*}

Como vemos, \(g_1\) tem um erro menor. Apesar de não chegar tão perto do ponto \((-1,-3)\), por exemplo, ela fica relativamente próxima da maioria, sendo assim uma melhor aproximação.

\subsection*{Adicional}

Gráfico das nossas aproximações (\(g_1\) em vermelho, \(g_2\) em azul):
\begin{center}
	\begin{tikzpicture}[scale=2]
		\begin{axis}
			[width=6cm,
				height=6cm,
				ticklabel style={font=\tiny},
				grid=both,
				grid style={line width=.1pt, draw=gray!10},
				major grid style={line width=.2pt,draw=gray!50},
				axis lines=middle,
				xmin=-7,xmax=7,ymin=-4,ymax=10,
				samples=400,
			]
            \addplot[red,domain=-10:10] {(1.1176*x^2)+(2*x)} node {\(g_1\)};
            \addplot[blue,domain=-10:10] {(2*x^2)-3} node {\(g_2\)};
			\node[circle,fill,inner sep=1pt] at (axis cs:-2,1) {};
			\node[circle,fill,inner sep=1pt] at (axis cs:-1,-3) {};
			\node[circle,fill,inner sep=1pt] at (axis cs:1,1) {};
			\node[circle,fill,inner sep=1pt] at (axis cs:2,9) {};
		\end{axis}
	\end{tikzpicture}
\end{center}


\section*{10.}
\subsection*{(a)}
Vamos montar a forma matricial:
\[
\begin{bmatrix}
    \Pi(x_1) \\
    \Pi(x_2) \\
    \Pi(x_3) \\
    \Pi(x_4)
\end{bmatrix}
=
\begin{bmatrix}
    1 & \frac{x_1}{\log_{10}{(x_1)}} \\
    1 & \frac{x_2}{\log_{10}{(x_2)}} \\
    1 & \frac{x_3}{\log_{10}{(x_3)}} \\
    1 & \frac{x_4}{\log_{10}{(x_4)}} \\
\end{bmatrix}
\cdot
\begin{bmatrix}
    a \\
    b
\end{bmatrix}
\]

Basta preencher com nossos valores:
\[
\begin{bmatrix}
    25 \\
    168 \\
    1229 \\
    9592
\end{bmatrix}
=
\begin{bmatrix}
    1 & 50 \\
    1 & 333.33 \\
    1 & 2500 \\
    1 & 20000 \\
\end{bmatrix}
\cdot
\begin{bmatrix}
    a \\
    b
\end{bmatrix}
\]

Resolvendo, temos:
\[
\begin{bmatrix}
    a \\
    b
\end{bmatrix}
=
\begin{bmatrix}
    12.82532 \\
     0.47907
\end{bmatrix}
\]

Que nos dá a função:
\[
    g(x) = 12.82532 + 0.47907 \frac{x}{\log{(x)}}
\]

\subsection*{(b)}

Basta usar \(x = 999999\) em nossa função:
\[
    \Pi(999999) \approx g(999999) \approx 79862
\]

\subsection*{Adicional}

Gráfico (escala logarítmica) da nossa aproximação, pontos usados, e aproximação de \(\Pi(999999)\):
\begin{center}
	\begin{tikzpicture}[scale=2]
		\begin{axis}
			[width=7cm,
				height=7cm,
				ticklabel style={font=\tiny},
				grid=both,
				grid style={line width=.1pt, draw=gray!10},
				major grid style={line width=.2pt,draw=gray!50},
                ymode=log,
                log basis y={10},
                xmode=log,
                log basis x={10},
			]
            \addplot[red,domain=25:1000000] {12.82532+(0.47907*(x/(ln(x)/ln(10))))};
			\node[circle,fill,inner sep=1pt] at (axis cs:100,25) {};
			\node[circle,fill,inner sep=1pt] at (axis cs:1000,168) {};
			\node[circle,fill,inner sep=1pt] at (axis cs:10000,1229) {};
			\node[circle,fill,inner sep=1pt] at (axis cs:100000,9592) {};
			\node[blue,circle,fill,inner sep=1pt] at (axis cs:999999,79862) {};
		\end{axis}
	\end{tikzpicture}
\end{center}

No geral, a aproximação parece ser muito boa. A ideia do exercício é muito interessante e mostra uma aplicação bem legal da aproximação. Partindo de alguns poucos pontos, conseguimos efetivamente modelar essa tendência em uma função muito simples.

Não tenho certeza de como comprovar sem contas, mas o gráfico me parece um bom indicativo da tendência.
\end{document}
