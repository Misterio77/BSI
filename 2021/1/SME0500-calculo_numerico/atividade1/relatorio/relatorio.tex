\documentclass[12pt]{article}
\usepackage{amsmath}
\usepackage{amssymb}
\usepackage{commath}
\usepackage{indentfirst}
\usepackage{mathtools}


\usepackage[portuguese]{babel}
\DeclarePairedDelimiter{\ceil}{\lceil}{\rceil}

\usepackage[small]{titlesec}
\titleformat{\part}[display]{\normalfont\large\bfseries}{\partname\ \thepart}{14pt}{\Large}
\renewcommand{\thesubsection}{\alph{subsection})} %chktex 9 chktex 10


\title{SME0500 \\ Cálculo Numérico \\ Atividade 1}
\author{Gabriel Silva Fontes | 10856803}
\begin{document}
\maketitle
\section*{4.}
\subsection{\(L\) e \(D\) em termos de \(A\)}
Os elementos de \(L\) e \(D\) podem ser escritos em termos dos elementos de \(A\) com as seguintes definições:
\begin{align*}
    D_j &= A_{jj} - \sum_{k=1}^{j-1}L^2_{jk} D_k \\
    L_{ij} &= \frac{1}{D_j} \left( A_{ij} - \sum_{k=1}^{j-1} L_{ik} L_{jk} D_k \right)
\end{align*}

\subsection{Relacionar Cholesky e LDLT}
Tomando uma decomposição LDLT, podemos relacionar à uma decomposição de Cholesky \(A = LL^T\) com essa relação:
\[
    A = LDL^T = LD^{1/2} {(D^{1/2})}^T L^T = LD^{1/2} {(LD^{1/2})}^T
\]

De forma análoga, é possível partir de uma decomposição \(A = HH^T\), tomando \(S\) como a matriz diagonal com a diagonal principal de \(H\):

\begin{align*}
    L &= HS^{-1} \\
    D &= S^2
\end{align*}

\subsection{Resolver \(Ax = b\)}
Vamos tomar o seguinte sistema (caso 1 do run.codes):
\begin{align*}
    A \cdot x &= b \\
    \begin{bmatrix}
        101 & 71 & 53 \\
         71 & 83 & 71 \\
         53 & 71 & 101
    \end{bmatrix}
    \begin{bmatrix}
        x_1 \\
        x_2 \\
        _3
    \end{bmatrix}
    &=
    \begin{bmatrix}
        1\\
        2\\
        3
    \end{bmatrix}
\end{align*}

Começando decompondo \(A\) em \(LDL^T\):
\begin{align*}
    A &= LDL^T \\
    A &=
    \begin{bmatrix}
        1 & 0 & 0 \\
        0.7030 & 1 & 0 \\
        0.5248 & 1.0197 & 1
    \end{bmatrix}
    \begin{bmatrix}
        101 & 0 & 0 \\
        0 & 33.089 & 0 \\
        0 & 0 & 38.779
    \end{bmatrix}
    \begin{bmatrix}
        1 & 0.7030 & 0.5248 \\
        0 & 1 & 1.0197 \\
        0 & 0 & 1
    \end{bmatrix}
\end{align*}

Daí, resolvemos \(Ly = b\) com substituição progressiva:
\begin{align*}
    Ly &= b \\
    \begin{bmatrix}
        101 & 0 & 0 \\
        0 & 33.089 & 0 \\
        0 & 0 & 38.779
    \end{bmatrix}
    \begin{bmatrix}
        y_1 \\
        y_2 \\
        y_3
    \end{bmatrix}
    &=
    \begin{bmatrix}
        1 \\
        2 \\
        3
    \end{bmatrix} \\
    y &=
    \begin{bmatrix}
        1 \\
        1.297 \\
        1.1526
    \end{bmatrix}
\end{align*}

Por fim, resolvemos \((D\cdot L^T)x = y\) com substituição regressiva:
\begin{align*}
    (D\cdot L^T)x &= y \\
    \left(\begin{bmatrix}
        101 & 0 & 0 \\
        0 & 33.089 & 0 \\
        0 & 0 & 38.779
    \end{bmatrix}
    \cdot
    \begin{bmatrix}
        1 & 0.703 & 0.5248 \\
        0 & 1 & 1.0197 \\
        0 & 0 & 1
    \end{bmatrix}
    \right)
    \begin{bmatrix}
        x_1 \\
        x_2 \\
        _3
    \end{bmatrix}
    &= 
    \begin{bmatrix}
        1 \\
        1.297 \\
        1.1526
    \end{bmatrix} \\
    x &=
    \begin{bmatrix}
        -0.0119 \\
         0.0089 \\
         0.0297
    \end{bmatrix}
\end{align*}

Que é o \(x\) que buscávamos.


\end{document} %  chktex 17
