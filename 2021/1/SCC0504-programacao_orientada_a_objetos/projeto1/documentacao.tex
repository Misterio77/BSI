\documentclass[12pt]{article}
\usepackage{booktabs}
\usepackage{indentfirst}
\usepackage{mathtools}
\usepackage{hyperref}
\usepackage{minted}

\usepackage[portuguese]{babel}

\usepackage[small]{titlesec}
\titleformat{\part}[display]
  {\normalfont\large\bfseries}{\partname\ \thepart}{14pt}{\Large}

\title{SCC0504 \\ Programação Orientada a Objetos \\ Trabalho 1}
\author{Gabriel Fontes | 10856803
    \and
    Vinicius Baca | 10788589
    }

\begin{document}
\maketitle

\section{Plataforma}
O projeto foi desenvolvido em Rust.

Por ser uma linguagem de baixo nível, algumas funcionalidades (presentes, por exemplo, na std do Java) precisam ser implementadas ou utilizada alguma dependência. Em especial a questão gráfica.

Optamos por utilizar um sistema ECS (Entity Control System) bem barebones para o projeto, este se chama Bevy. O Bevy contém algumas ferramentas que seriam trabalhosas de se implementar corretamente no Rust, como, por exemplo, a questão gráfica (janelas, renderização de sprites), um event loop (threaded), sincronização de estado mutável (também threaded), e levantamento de eventos.

Fora esses componentes (que são muito menos que os equivalentes fornecidos em Java), toda a lógica e técnica do jogo foi implementada do zero, sem ferramentas externas.

Num sistema ECS, é preferível agregação à composição e herança. No Rust também é bastante comum utilizar Enums (além dos Traits, que são interfaces) para atingir polimorfismo, técnica bastante empregada no Bevy e no nosso jogo.

As sprites foram extraídas do jogo com captura de tela, e re-desenhadas em 8x8 ou 16x16 pixels. Algumas delas são sheets (para animações e direções diferentes), e algumas também foram dessaturadas (para reutilização da mesma sprite para várias cores).

\section{Instruções}
\subsection{Cargo}
Caso não tenha, instale com o apt (no caso do Ubuntu):
\mint{bash}|sudo apt install cargo|

\subsection{Compilação}
Basta entrar dentro da pasta `codigo', e utilizar o cargo:
\mint{bash}|cargo build --release|
Pode demorar um ou dois minutos, por ter algumas dependências.

\subsection{Execução}
Também com o cargo, dentro da pasta `código':
\mint{bash}|cargo run --release|

\section{Esclarecimentos}
O jogo segue fielmente as mecânicas do Skooter (em especial o cooldown de movimento), mas não conseguimos implementar a animação gradual de movimento entre posições. Mesmo assim, é possível prever o movimento dos robôs se baseando na direção em que estão olhando.

\end{document}
