\documentclass[12pt]{article}
\usepackage{indentfirst}

\usepackage[portuguese]{babel}

\usepackage[tiny]{titlesec}
\titleformat{\part}[display]{\normalfont\large\bfseries}{\partname\ \thepart}{14pt}{\Large}

\title{SCC0540 \\ Bancos de Dados}
\author{Gabriel Silva Fontes | 10856803}
\begin{document}
\maketitle

\section{Quais os problemas em implementar persistência de dados utilizando um sistema de arquivos?}
Em suma, desacoplar funcionamento dos arquivos da modelagem de dados.

\section{Como os SGBDs garantem independência de dados?}

\section{Cite e explique alguns requisitos importantes dos SGBDs.}

\section{Explique independência física e lógica de dados no contexto da arquitetura em três níveis de esquema (three-schema) dos sistemas de bancos de dados.}

\section{Explique o ciclo de vida de Sistemas de Banco de Dados no que diz respeito ao projeto da base de dados.}

\section{Por que o interpretador/compilador de consultas é nescessário e o que ele faz?}

\section{Discuta e exemplifique abstração de dados.}

\section{Explique e relacione os conceitos de esquema e instância da base de dados.}

\section{O que é modelagem conceitual e qual sua importância?}

\section{Cite alguma aplicações que usam banco de dados que você acha interessantes.}


\maketitle
\end{document}
