\documentclass{article}

\usepackage{amsmath}
\usepackage{amssymb}
\usepackage{booktabs}

\newcommand{\bftab}{\fontseries{b}\selectfont}

\title{SMA0508 - Matemática Discreta \\ Atividade 3 (Parte 2)}
\author{Gabriel Fontes - 10856803}

\begin{document}
\maketitle
Como enunciado, temos $A = \{p_1, ..., p_k\}$, um conjunto de primos positivos. Além disso, temos $n = p_1...p_k+1$.
\medskip

Queremos provar que existe um primo positivo $p$ tal que $p \notin A$.
Como $n-1$ é composto por um ou mais números primos (que são, por definição, $>1$), sabemos que $n > 2$. Dessa maneira, pela definição de números primos, $n$ deve ser, exclusivamente, um número primo ou um número composto.
\medskip

Todo número
Para o caso de $n$ ser primo, claramente existe um primo $p$ que divide $n$, que é ele mesmo. Suponha por absurdo que $p \in A$. Por consequência, $p$ obrigatoriamente divide $p_1...p_k$. Para que $p$ divida $p_1...p_k$ e $n$ ao mesmo tempo, ele também precisa dividir $1$, coisa que não é possível para um primo, absurdo.  Provamos assim que, para $n$ primo, existe um primo $p$ tal que $p \notin A$.

Para o caso de $n$ ser um número composto, existe um primo $p$ que o divide. Supondo por absurdo que $p \in A$, $p$ obrigatoriamente divide $p_1...p_k$. Já que $p$ divide $n$ e também 
\end{document}
