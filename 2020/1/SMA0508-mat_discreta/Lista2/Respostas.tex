\documentclass{article}

\usepackage{amsmath}
\usepackage{amssymb}
\usepackage{booktabs}

\newcommand{\bftab}{\fontseries{b}\selectfont}

\title{SMA0508 - Matemática Discreta \\ Atividade 2}
\author{Gabriel Fontes - 10856803}

\begin{document}
\maketitle

\section{}
Precisamos obter $mdc(94735, 245)$ aplicando o método de Euclides:
$$\begin{tabular}{ l | c | c | c | r}
        & 386 & 1   & 2  & 16        \\
  \hline
  94735 & 245 & 165 & 80 & \bftab 5  \\
  \hline
  165   & 80  & 5   & 0  &           \\
\end{tabular}$$
$$\therefore mdc(94735, 245) = 5$$

\section{}
Vamos escrever $mdc(94735, 245)$ como combinação linear de $94735$ e $245$. \\
Começando pela última divisão com resto $\ne 0$ temos:
$$165 = 80 \cdot 2 + 5$$
Escrevendo o resto como combinação linear do dividendo e quociente:
$$5 = 1 \cdot 165 - 2 \cdot 80$$
Agora basta substituir por combinações lineares e simplificar sucessivamente, até chegar no divisor e quociente originais:
\begin{align*}
5 &= 1 \cdot 165 - 2 \cdot (1 \cdot 245 - 1 \cdot 165)   \\
  &= 3 \cdot 165 - 2 \cdot 245                       \\
  &= 3 \cdot (1 \cdot 94735 - 386*245) - 2 \cdot 245   \\
  &= \mathbf{3} \cdot 94735 -\mathbf{1160} \cdot 245
\end{align*}
Sendo essa a combinação linear que queríamos encontrar.
\begin{align*}
\therefore r &= 5 \\
s &= 1160
\end{align*}

\section{}
Supondo um $d \in \mathbb{Z}$ tal que $ d|a$ e $d|b$. Vamos lembrar da definição de "divide", $\exists x \in \mathbb{Z}$ se e somente se:
$$x \cdot d = a$$
$$x \cdot d = b$$
Agora vamos tomar $r$ e $s$ previamente enunciados, lembrando que $r, s \in \mathbb{Z}$. Pela definição de "divide", podemos afirmar que um número divisível por outro continuará divisível se multiplicado por um inteiro (já que $x$ é um inteiro qualquer). Sendo assim:
$$d|a \implies d | r \cdot a$$
$$d|b \implies d | s \cdot b$$
Um número que divide dois números também divide a soma desses números, ou seja:
$$d |(r \cdot a + s \cdot b)$$
Já vimos que $r \cdot a + s \cdot b = 1$, então sabemos que:
$$d|1$$
Voltando a definição, temos:
$$x \cdot d = 1$$
A única maneira de uma multiplicação entre dois inteiros resultar em $1$ é se ambos forem $1$ (ou $-1$).
Sendo assim, $d = 1$. A única maneira de $mdc(a,b) = 1$ é se $a$ e $b$ forem primos.

\section{}

\end{document}
