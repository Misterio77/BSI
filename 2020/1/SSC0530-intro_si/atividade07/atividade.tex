\documentclass[12pt]{article}
\usepackage[sfdefault]{FiraSans}
\usepackage{hyperref}
\hypersetup{
  colorlinks   = true, %Colours links instead of ugly boxes
  urlcolor     = blue, %Colour for external hyperlinks
  linkcolor    = blue, %Colour of internal links
  citecolor    = blue %Colour of citations
}


\usepackage[num]{abntex2cite}
\usepackage[portuguese]{babel}
\usepackage[small]{titlesec} %Diminuir tamanho das sections
\usepackage{indentfirst} %Identar os primeiros parágrafos em cada seção
\sloppy

\title{SCC0530 \\Introdução a Sistemas de Informação \\ Atividade 7}
\author{Gabriel Fontes - 10856803}

\begin{document}
\maketitle
\section{Qual o problema da Graybar descrito nese caso? Como esse problema afetava seu desempenho dos negócios?}
A empresa precisava de preços e atendimento ao cliente mais competitivos, a fim de continuar relevante e forte no ramo. Como distribuidor atacadista, a empresa depende de grandes transações para realmente lucrar, e nem todos os clientes tinham o perfil de grandes e constantes compradores, e esses deveriam ser priorizados, atraídos e convertidos.
\section{Quais questões humanas, organizacionais e tecnológicas tiveram que ser consideradas pela empresa a fim de desenvolver uma solução para a análise de clientes?}
Grande quantidade de dados legado, dificuldade de identificar clientes rentáveis, e sistemas pouco visuais.
\section{Como o CRM analítico mudou a maneira da Graybar conduzir seu negócio? Compare a maneira com que ela tratava seus relacionamentos com os clientes antes e depois da implementação do CRM analítico.}
Tradicionalmente, a empresa tinha foco nos clientes com maiores gastos. Com a estratificação de clientes, focar esforços de vendas e conversão ficou mais fácil e direto. Unindo a dados em tempo real, decisões ficaram mais eficientes e certeiras.
\section{Dê exemplos de três decisões que foram melhoradas pelo novo sistema de análise de clientes.}
A decisão de quais clientes focar esforços e verbas. A decisão de focar em clientes não nescessariamente com maiores gastos, mas os mais rentáveis. A divisão dos clientes nos quatro tipos.
\end{document}
