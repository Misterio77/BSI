\documentclass[12pt]{article}
\usepackage[sfdefault]{FiraSans}
\usepackage{hyperref}
\hypersetup{
  colorlinks   = true, %Colours links instead of ugly boxes
  urlcolor     = blue, %Colour for external hyperlinks
  linkcolor    = blue, %Colour of internal links
  citecolor    = blue %Colour of citations
}

\usepackage[num]{abntex2cite}
\usepackage[portuguese]{babel}
\usepackage[small]{titlesec} %Diminuir tamanho das sections
\usepackage{indentfirst} %Identar os primeiros parágrafos em cada seção


\title{SCC0530 \\Introdução a Sistemas de Informação \\ Atividade 5}
\author{Gabriel Fontes - 10856803}

\begin{document}
\maketitle
\section{Aplique os modelos de cadeia de valor e de forças competitivas ao eBay.}
O modelo de atividades do eBay inclui primariamente o desenvolvimento, manutenção e operação de plataforma web, mobile, e infraestrutura interna de TI\cite{ebay_info}. Por ter sido criado e desenvolvido para facilitar comércio \textit{consumer-to-consumer} e \textit{business-to-consumer} (varejo), o eBay se restringe a negociações de terceiros, anunciando e concretizando vendas.

As forças competitivas ao eBay incluem outros sites para concretização de venda, competidores varejistas, e a gradual diminuição da demanda por leilões (problema mitiagdo pela diversificação das atividades da empresa).

\section{Qual o modelo de negócios e a estratégia de negócios do eBay? O quão bem-sucedidos eles têm sido?}
Inicialmente o serviço se restringia aos tradicionais leilões, modalidade pela qual a empresa é mais conhecida. Conforme o eBay efetuou aquisições e expansões, diversificou para incluir compras por varejo (o chamado "Comprar já!"), venda e comparação de objetos fixos (Por ISBN, UPC, etc) e classificados.

A empresa obteve muito sucesso com seu serviço de leilões, se tornando um dos sites mais tradicionais nesse campo, e um exemplo notório de sucesso da "bolha da Internet"\cite{ebay_bolha}. Com o capital, investiu em aquisições e diversificou operações.

\section{Quais os problemas frequentemente enfrentados pelo eBay? Como a empresa está tentando resolvê-los?}
A decrescente rentabilidade do mercado de leilões online, dificuldade de inserção no mercado de varejo e aquisições das quais se tirou pouco proveito.

A empresa vem tentando se inserir no mercado de varejo, com estratégias mais agressivas com estímulos aos varejistas. O eBay também possui um grande histórico de aquisições.
\section{As soluções implantadas são boas? Explique. Existem outras soluções que o eBay deveria considerar?}
As estratégias para estimular varejistas tiveram um certo sucesso, mas não o suficiente para vencer um dos gigantescos competidores. Além disso, a empresa tem afastado principal nicho do eBay: leilões, contribuindo com a queda de visibilidade e rentabilidade destes no site\cite{ebay_leiloes}.

Quanto a sua estratégia de aquisições, o eBay, em grande parte, falhou em integrar essas ao seu produto principal, e até mesmo umas às outras. O conceituado e bem-sucedido PayPal, um dos líderes no setor\cite{paypal_lider}, raramente foi integrado ou atrelado à empresa que, até 2015\cite{paypal_spinnof}, foi sua proprietária, sendo utilizado meramente como método de pagamento. O Skype, programa de telecomunicações que o grande público sequer lembra que já foi do eBay, foi adquirido\cite{compra_skype} e depois vendido\cite{venda_skype}, não utilizado para agregar valor ao eBay ou seus subsidiários.

Por mais que o mercado de varejo seja muito grande e atrativo, o eBay tem sua tradição nos leilões, que vem perdendo visibilidade. Para evitar a migração desses clientes a outros sites (como craiglist), a empresa deveria manter sessões dedicadas do site à essa fatia de negócios, com uma clara separação entre eles e varejistas.
\section{Qual o papel dos fatores pessoais, organizacionais e tecnológicos na resposta do eBay aos problemas?}
A eBay é um exemplo de empresa que opera puramente com serviços de TI. Tecnologia e organização é essencial em toda decisão empresarial tomada dentro de uma empresa desse tipo.

Fatores pessoais dos executivos podem ter sido relevantes nas decisões de distanciamento dos leilões, visto que houveram grandes mudanças na direção da empresa.
\section{O eBay alcançará o sucesso no longo prazo? Justifique.}
O recente surto do coronavírus evidencia uma vantagem logística do eBay (e outros sites que concretizam vendas de terceiros) sob competidores com estoque e logística própria\cite{amazon_preju}.

Com a paralização de lojas físicas e uma demanda (ainda) maior por e-commerce, muitos comerciantes de (porte pequeno à médio) passam a buscar uma maneira de vender online. Com as dificuldades em lançar e divulgar uma loja online própria, esse tipo de serviço intermediador tem ótimas oportunidades\cite{coronavirus_ebay}.

Com as decisões corretas, e com cuidado para não perder o seu nicho principal, o eBay pode crescer e expandir seus negócios, em especial como anunciante e fulfiller de negociações.
\bibliography{bibliografia}
\end{document}
