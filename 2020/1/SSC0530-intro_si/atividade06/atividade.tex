\documentclass[12pt]{article}
\usepackage[sfdefault]{FiraSans}
\usepackage{hyperref}
\hypersetup{
  colorlinks   = true, %Colours links instead of ugly boxes
  urlcolor     = blue, %Colour for external hyperlinks
  linkcolor    = blue, %Colour of internal links
  citecolor    = blue %Colour of citations
}


\usepackage[num]{abntex2cite}
\usepackage[portuguese]{babel}
\usepackage[small]{titlesec} %Diminuir tamanho das sections
\usepackage{indentfirst} %Identar os primeiros parágrafos em cada seção
\sloppy

\title{SCC0530 \\Introdução a Sistemas de Informação \\ Atividade 6}
\author{Gabriel Fontes - 10856803}

\begin{document}
\maketitle
\section{Identifique e descreva o problema discutido nesse caso. Quais fatores humanos, organizacionais e tecnológicos contribuíram para o problema?}
A Vodafone, como uma empresa global, sofria com uma grande fragmentação operacional em suas subsidiárias, e grande quantidade de sistemas legado pouco integrados entre si.

Ao longo do seu processo de crescimento, a Vodafone estabeleceu ou adquiriu diversas operadoras espalhadas pelo mundo. Esse crescimento, que ocorreu de forma relativamente independente em cada subsidiário, deixou um legado de fragmentação difícil de se quebrar. 


\section{Por que a Vodafone teve que gastar tanto tempo lidando com a mudança durante a sua transformação organizacional?}
Por sua abrangência global, existem grandes diferenças culturais e operacionais em diferentes mercados, nescessitando soluções personalizadas em cada ramo da empresa. Muitos sistemas legado tiveram que ser substituidos, e seus usuários nem sempre tiveram uma transição fácil. 


\section{Por que uma sistema ERP foi nescessário para a transformação organizacional global da empresa?}
Para poder finalmente utilizar seu poder estratégico e financeiro como uma empresa global, a Vodafone precisava de sistemas unificados. Desfragmentando a empresa e centralizando as informações, estas tem muito mais utilidade para a empresa como um todo.

\section{Quais questões humanas, organizacionais e tecnológicas tiveram que ser consideradas pela equipe do projeto para assegurar que a transformação fosse bem-sucedida?}
Os diferentes históricos e culturas locais de cada subsidiária foram grandes desafios para a empresa. Além disso, muitos sistemas legado e funcionários antigos tornam a implatanção um processo ainda mais delicado. Foi nescessário acompanhamento, treinamento e uma transição gradual nessa mudança.

\section{Quais foram os benefícios da transformação organizacional global da Vodafone? Como ela alterou a tomada de decisões e a forma como a empresa operava?}
Aumento da eficiência organizacional, redução de custos, maior poder de barganha, sistemas mais intuitivos e consistentes, além de maior poder estratégico e uniformização dos serviços e sistemas. As decisões agora podem ser feitas levando em consideração informações, que até então levavam tempo para serem compiladas, em tempo real.

%\bibliography{bibliografia}
\end{document}
