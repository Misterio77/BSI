\documentclass[12pt]{article}
\usepackage{amsmath}
\usepackage{amssymb}
\usepackage{commath}
\usepackage{systeme}
\usepackage{booktabs}
\usepackage{indentfirst}
\usepackage{mathtools}
\usepackage{tikz}
\usepackage[portuguese]{babel}
\DeclarePairedDelimiter{\ceil}{\lceil}{\rceil}

\usepackage[small]{titlesec}
\titleformat{\part}[display]
  {\normalfont\large\bfseries}{\partname\ \thepart}{14pt}{\Large}

\renewcommand\thesubsection{\alph{subsection})}% chktex 9 chktex 10
\title{SMA0501 - Cálculo I \\ P2 - Parte 1}% chktex 8
\author{Gabriel Fontes - 10856803}% chktex 8

\begin{document}
\maketitle
Eu, Gabriel Silva Fontes, estou apresentando, a seguir, minhas próprias soluções para as atividades propostas na P2 - PARTE I. % chktex 8
Atesto que me dediquei as mesmas, seja individualmente, seja em colaboração com outro/as colegas.
Nos casos em que a discussão com colegas tenha resultado em soluções descritas de modo semelhante às de outro/as aluno/as, atesto que entendo a resolução apresentada no material individual que estou entregando.

\section{}
\[
    \lim_{x\rightarrow 0}(f(x)g(x)+\cos x)
    = \lim_{x\rightarrow 0}f(x) \cdot \lim_{x\rightarrow 0}g(x) + \lim_{x\rightarrow 0}\cos x \\
\]
Calculando o limite de \(f(x)\):
\begin{align*}
    & \lim_{x\rightarrow 0}\abs{\sin{x}} \leq \lim_{x\rightarrow 0}f(x) \leq \lim_{x\rightarrow 0}3\abs{x} \\
    &\implies \abs{\sin{0}} \leq \lim_{x\rightarrow 0}f(x) \leq 3\abs{0} \\
    &\implies 0 \leq \lim_{x\rightarrow 0}f(x) \leq 0 \\
    &\implies \lim_{x\rightarrow 0}f(x) = 0
\end{align*}
Para podermos multiplicar \(g(x)\) por \(f(x) = 0\), vamos verificar o seu limite (em especial, se é limitado):
\begin{align*}
    & 0 \leq \lim_{x\rightarrow 0} g(x) \leq \lim_{x\rightarrow 0} (1 + \abs{\sin x}) \\
    &\implies 0 \leq \lim_{x\rightarrow 0} g(x) \leq 1
\end{align*}
Dessa forma, podemos afirmar, pelo teorema do confronto:
\begin{align*}
    \lim_{x\rightarrow 0} f(x)\cdot g(x)
    &= \lim_{x\rightarrow 0} f(x) \cdot \lim_{x\rightarrow 0} g(x) \\
    &= 0 \cdot \lim_{x\rightarrow 0} g(x) \\
    &= 0
\end{align*}
Substituindo no enunciado:
\begin{align*}
    \lim_{x\rightarrow 0} (f(x)\cdot g(x) + \cos{x})
    &= \lim_{x\rightarrow 0} \cos{x} \\
    &= \cos{0} \\
    &= 1
\end{align*}

\section{}
O erro ocorre na multiplicação em \(\lim_{x\rightarrow +\infty} (x\cdot0)\), pois não é correto operar em uma indeterminação, isto é, o seguinte está incorreto:
\begin{align*}
    \lim_{x\rightarrow +\infty} (x\cdot0)
    &= \lim_{x\rightarrow +\infty} x \cdot \lim_{x\rightarrow +\infty} 0\\
    &= \infty \cdot 0 \\
    &= 0
\end{align*}

Calculando da maneira correta:
\begin{align*}
    \lim_{x\rightarrow +\infty} \sqrt{x^2+x}-x
    &= \lim_{x\rightarrow +\infty} \sqrt{x^2+x}-x \cdot \lim_{x\rightarrow +\infty} \frac{\sqrt{x^2+x}+x}{\sqrt{x^2+x}+x} \\
    &= \lim_{x\rightarrow +\infty} \frac{x}{\sqrt{x^2+x}+x} \\
    &= \lim_{x\rightarrow +\infty} \frac{1}{\sqrt{1+\frac{1}{x}}+1} \\
    &= \frac{\lim_{x\rightarrow +\infty} 1}{\lim_{x\rightarrow +\infty} \sqrt{1+\frac{1}{x}}+1} \\
    &= \frac{1}{2}
\end{align*}

\section{}
\subsection{}
\begin{align*}
    \lim_{x\rightarrow 2} \frac{f(x)}{x}
    &= \lim_{x\rightarrow 2} \frac{f(x)}{x} \cdot \lim_{x\rightarrow 2} \frac{x}{x} \\
    &= \lim_{x\rightarrow 2} \frac{f(x)}{x} \cdot \lim_{x\rightarrow 2} \frac{1}{x} \cdot \lim_{x\rightarrow 2} x \\
    &= \lim_{x\rightarrow 2} \frac{f(x)}{x^2} \cdot \lim_{x\rightarrow 2} x \\
    &= 1 \cdot 2 = 2
\end{align*}
\subsection{}
\begin{align*}
    \lim_{x\rightarrow 0} f(x)
    &= \lim_{x\rightarrow 0} f(x) \cdot \lim_{x\rightarrow 0} \frac{x}{x} \\
    &= \lim_{x\rightarrow 0} f(x) \cdot \lim_{x\rightarrow 0} \frac{1}{x} \cdot \lim_{x\rightarrow 0} x \\
    &= \lim_{x\rightarrow 0} \frac{f(x)}{x} \cdot \lim_{x\rightarrow 0} x \\
    &= 0 \cdot 0 = 0
\end{align*}
\subsection{}
\begin{align*}
    \lim_{x\rightarrow +\infty} f(x)
    &= \lim_{x\rightarrow +\infty} f(x) \cdot \lim_{x\rightarrow +\infty} \frac{x^2+x}{x^2+x} \\
    &= \lim_{x\rightarrow +\infty} f(x) \cdot \lim_{x\rightarrow +\infty} \frac{1}{x^2+x} \cdot \lim_{x\rightarrow +\infty} x^2+x \\
    &= \lim_{x\rightarrow +\infty} \frac{f(x)}{x^2+x} \cdot \lim_{x\rightarrow +\infty} x^2+x \\
    &= +\infty
\end{align*}

\end{document}
