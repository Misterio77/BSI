\documentclass[12pt]{article}

\usepackage{amsmath}
\usepackage{amssymb}
\usepackage{commath}
\usepackage{booktabs}
\usepackage{indentfirst}
\usepackage[portuguese]{babel}

%Reduce part, section and subsection font sizes
\usepackage[small]{titlesec}
\titleformat{\part}[display]
  {\normalfont\large\bfseries}{\partname\ \thepart}{14pt}{\Large}

\title{SMA0501 - Cálculo I \\ ATR 1}% chktex 8
\author{Gabriel Fontes - 10856803}% chktex 8

\begin{document}
\maketitle

\part*{Aula I}
\section{Números reais prof. Régis}
\subsection{Identifique se os dois objetos são iguais}
Sim. Por convergir em \(0\):
\[ (1/2,1/4,1/8,\dots) = (0,0,0,\dots) \]
Podemos afirmar:
\[ (2+1/2, 2+1/4, 2+1/8, \dots) = (0+2,0+2,0+2,\dots) \]
\[ \therefore (2+1/2, 2+1/4, 2+1/8, \dots) = (2,2,2,\dots) \]

\subsection{Crie dois pares de novos objetos}
\[ (1/3;1/3;1/3;\dots) = (0,3; 0,33; 0,333; \dots) \]
Pois ambos convergem em \( 1/3 \) (\(0,333\dots\)).

\[ (0,9;0,99;0,999;\dots) \ne (0,1;0,01;0,001;\dots) \]
Pois um converge em \(1\), e o outro em \(0\).

\section{Desigualdade modular prof. Aurichi}
\subsection{Enuncie e escreva as resoluções das desigualdades modulares}
\subsubsection{\(\abs{x-7} < 3\)}
O professor Aurichi fez o exercícios de duas formas, a primeira se usando da propriedade \(\abs{x} \leq a \iff -a \leq x \leq a\):
\[\abs{x-7} < 3 \implies -3 < x - 7 < 3\]
\[
	\boxed{
		4 < x < 10
	}
\]

\bigskip
\bigskip

E a segunda forma, estudando o sinal do módulo. Primeiro para \(x-7 \geq 0 \):
\[
	x-7 \geq 0 \implies x \geq 7
\]
Neste caso \( \abs{x-7} = x-7 \), logo:
\[
	x-7 < 3 \implies x < 10
\]
\[
	\therefore 7 \leq x < 10
\]

\smallskip

Agora para \(x-7 < 0\):
\[
	x-7 < 0 \implies x < 7
\]
Neste caso \(\abs{x-7} = -x+7\), logo:
\[
	-x+7 < 3 \implies x > 4
\]
\[
	\therefore 4 < x < 7
\]

Unindo esses dois intervalos, temos que a equação \(\abs{x-7} < 3\) é satisfeita quando:
\[
	\boxed{
		4 < x < 10
	}
\]

\subsubsection{\( \abs{2x+1}+\abs{x-1}<3 \)}
Os módulos trocam de comportamento individualmente em seus pontos críticos, que são \(x=-\frac{1}{2}\) e \(x=1\). Começamos com o caso \(x < -\frac{1}{2}\):
\[
	-(2x+1) - (x-1) < 3 \implies x > -1
\]
\[
	\therefore -1 < x < -\frac{1}{2} = (-1,-\frac{1}{2})
\]

Agora o caso \(-\frac{1}{2} \leq x < 1\):
\[
	2x+1 + 1 - x< 3 \implies x < 1
\]
\[
	\therefore -\frac{1}{2} \leq x < 1 = [-\frac{1}{2},1)
\]

Agora o caso \(x \geq 1 \):
\[
	2x+1 + x-1 < 3 \implies x < 1
\]
\[
	\therefore \varnothing
\]

Unindo todos os intervalos, temos que os \(x\) que satisfazem a inequação estão no intervalo:
\[
	\boxed{
		(-1,1)
	}
\]
\subsection{Escreva os intervalos que representam dados os subconjuntos \(A\) e \(B\), e encontre um \(r\) para dada inclusão}
Subconjunto \(A = \{x \in \mathbb{R} : 4x-3<5x+2\}\):
\begin{align*}
	4x-3 &< 5x+2 \\
	-5 &< x
\end{align*}

Que pode então ser escrito como:
\[
	\boxed{
		A = \{x \in \mathbb{R} : -5 < x\} = (-5,\infty)
	}
\]

\bigskip \bigskip

Agora, o subconjunto \(B = \{x \in \mathbb{R} : \abs{x}<1\} \):
\[
	\abs{x} < 1
\]
\[
	-1 < x < 1
\]

Que pode então ser escrito como:
\[ 
	\boxed{
		B = \{x\in\mathbb{R} : -1 < x < 1\} = (-1,1)
	}
\]

\bigskip \bigskip \bigskip

Para que \( (4-r,4+r) \subset (2,5) \), precisamos que:
\[ 2 \leq 4 + r \leq 5 \text{ e também } 2 \leq 4 - r \leq 5 \]

Sendo assim:
\[ -2 \leq r \leq 1 \text{ e } -1 \leq r \leq 2 \]

ou seja:
\[
	\boxed{
		-1 \leq r \leq 1
	}
\]

\part*{Aula II}
\section{Números Reais prof. Lymberopoulos}
\subsection{Prove que \(\sqrt{3} \notin \mathbb{Q}\)}
Vamos usar o mesmo método que o professor Lymberopoulos usou para provar que \(\sqrt{2} \notin \mathbb{Q} \)

Suponha por contradição que \(\sqrt{3} \in \mathbb{Q}\). Parte disso que \(\sqrt{3} = \frac{a}{b}\), sendo \(a\) e \(b\) primos entre si.
Sendo assim, \(3 = \frac{a^2}{b^2}\) e \(3b^2 = a^2\). Por isso, \(a^2\) deve ser divisível por \(3\), porém \(a\) também deve (teorema fundamental da aritmética). Assim temos que \(b^2 = 3k^2\). Assim, \(b^2\), e também \(b\) precisam ser divisíveis por 3, que significa que \(a\) e \(b\) não são primos entre si. Contradição.
\[
	\boxed{
		\therefore \sqrt{3} \notin \mathbb{Q}
	}
\]
\subsection{Dê um exemplo númerico da Propriedade Arquimediana}
Escolherei os valores \( x = 10\), \(y = 3\).

Pela propriedade arquimediana, existe \(n \in \mathbb{N}: ny > x \).
Podemos, por exemplo, usar \(n = 4\), que implica em:
\[ 4*3 > 10 \]
\[ 12 > 10 \]

\end{document}
