\documentclass[12pt]{article}

\usepackage{amsmath}
\usepackage{amssymb}
\usepackage{commath}
\usepackage{systeme}
\usepackage{booktabs}
\usepackage{indentfirst}
\usepackage[portuguese]{babel}

%Reduce part, section and subsection font sizes
\usepackage[small]{titlesec}
\titleformat{\part}[display]
  {\normalfont\large\bfseries}{\partname\ \thepart}{14pt}{\Large}

\title{SMA0501 - Cálculo I \\ ATR 3}% chktex 8
\author{Gabriel Fontes - 10856803}% chktex 8

\begin{document}
\maketitle

\section{}
\subsection{}
Como o próprio nome diz, os problemas no cálculo diferencial e integral são a diferenciação e a integração.

A derivada de uma função é o que descreve o grau de variação dessa função, isto é, sua taxa de variação ou a aproximação linear dos pontos. Isso pode ser visto geometricamente como a inclinação da reta tangente de cada ponto no gráfico da função.

O segundo objeto de estudo é a chamada integral. A integral definida pode ser usada para descrever áreas (e também volumes), por meio de combinações infinitesimais. A integral definida de uma função pode ser vista geometricamente como a área delimitada pelo seu gráfico.
\subsection{}
Na função \(f(x) = x^3\), vamos tomar os pontos \((2,8)\) e \((x,x^3)\) para construir uma secante \(r\).

\(r\) tem coeficiente angular:
\begin{align*}
	m &= \frac{x^3-8}{x-2} = \frac{(x-2)\cdot(x^2 + 2x + 2^2)}{x-2} \\
	  &= x^2 + 2x + 2^2
\end{align*}
Com \(x\rightarrow2\), teremos \(m = 12\) e portanto:
\[
	y - 8 = 12\cdot(x-2)
\]
\[
	\boxed{
		y = 12x - 16
	}
\]
\section{}
\subsection{}
O chamado teorema fundamental do cálculo relaciona as duas operações. Uma das antiderivadas (ou primitivas) de uma função, quando existe, pode ser utilizada para calcular a integral definida. Esta antiderivada também é chamada de integral indefinida.

\subsection{}
TODO

\section{}

\section{}
\subsection{}
Falsa. O valor do limite é dado pela aproximação pela direita e pela esquerda na função. Poderíamos ter, perfeitamente, a função:
\[
f(x) = \frac{x}{x-1}
\]
Que não está definida em \(x=1\) (logo seu limite não pode ser igual a \(f(1 )\)).

Ou até mesmo, a função:
\[
	f(x) = \systeme*{x\text{, } x \neq 1, 2x\text{, } x = 1}
\]
Que está definida no ponto, porém cujo valor não coincide com o limite da função tendendo a esse ponto.

\subsection{}
Sim. Quando uma função não é contínua em um ponto (possivelmente por não estar definida nele), porém seu limite nesse mesmo ponto existe (e tornaria a função contínua caso fizesse parte dela), temos uma descontinuidade removível.
Um exemplo de função não definida no ponto \(x=0\) é:
\[
	f(x) = \frac{x}{x^2}
\]
Cujo limite é:
\[
	\lim_{x \rightarrow 0} \frac{x}{x^2} = x = 0
\]
\subsection{}
Intuitivamente, uma função é contínua se pequenos incrementos (ou decrementos) na entrada geram (também pequenos) incrementos e decrementos na saída. Também pode ser vista como contínua em um ponto \(p\) quando:
\[
	\lim_{x \rightarrow p} = f(p)
\]
Ou seja, três condições devem ser satisfeitas: \(f\) precisa estar definida em \(p\) (\(p \in Dom(f)\)), o limite precisa existir, e o limite precisa ser igual à \(f(c)\).
\bigskip

Uma maneira mais formal de continuidade é a definição de Weierstrass e Jordan (épsilon e delta). Nessa definição, \(f\) é contínua em \(p\) se:
\[
	\text{dado } \epsilon > 0\text{, } \exists \delta > 0: \abs{x-p} < \delta \implies \abs{f(x) - f(p)} < \epsilon
\]
Isto é, para um \(x\) abitrariamente próximo de \(p\), \(f(x)\) também obrigatoriamente precisará estar próximo de \(f(p)\). Coisa que não acontece com um ponto não contínuo da função, já que o \(f(x)\) pode ``cair fora'' do intervalo delimitado por \(\epsilon\).

\subsection{}
TODO 
\section{}
\subsection{}
TODO
\subsection{}
TODO
\subsection{}
TODO
\subsection{}
TODO
\end{document} 
