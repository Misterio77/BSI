\documentclass[12pt]{article}
    \usepackage{amsmath}
    \usepackage{amsthm}
    \usepackage{amssymb}
    \usepackage{commath}
    \usepackage{systeme}
    \usepackage{booktabs}
    \usepackage{indentfirst}
    \usepackage{mathtools}
    \usepackage[portuguese]{babel}
    \usepackage{pgfplots}
    \usepackage{tikz}
    \pgfplotsset{compat=1.17}
    \DeclarePairedDelimiter{\ceil}{\lceil}{\rceil}
    %Reduce part, section and subsection font sizes
    \usepackage[small]{titlesec}
    \titleformat{\part}[display]
      {\normalfont\large\bfseries}{\partname\ \thepart}{14pt}{\Large}
    \usepackage{xpatch}
    \renewcommand\thesubsubsection{\alph{subsubsection})}% chktex 9 chktex 10
    \theoremstyle{definition}
    \newtheorem*{definition}{Def}

\title{SMA0501 - Cálculo I \\ ATR 7}% chktex 8
\author{Gabriel Fontes - 10856803}% chktex 8

\begin{document}
\maketitle
\part{Aula X Parte I}
\setcounter{section}{0}
\section{Assista a aula}

\section{}
\subsection{Elenque os dois limites e mostre que são iguais}
\begin{align*}
    x &:= p+h \\
    \implies x-p &= h \\
\end{align*}
Também sabendo que \(x\rightarrow p \implies x-p \rightarrow 0\), agora basta substituir:
\begin{align*}
    \lim_{h\rightarrow0}\frac{f(p+h)-f(p)}{h}
    &= \lim_{x\rightarrow p}\frac{f(x)-f(p)}{x-p} \\
    &= \lim_{x-p\rightarrow 0}\frac{f(x)-f(p)}{x-p} \\
    &= \lim_{h\rightarrow 0}\frac{f(p+h)-f(p)}{h} \\
\end{align*}
\subsection{\(f(x) = \abs{x-1}\) não possui derivada em \(p=1\), justifique}
\(f\) é contínua em \(p = 1\). Vamos verificar se é derivável nesse mesmo ponto, calculando o limite:
\begin{align*}
    \lim_{x\rightarrow 1}\frac{f(x)-f(1)}{x-1}
    &= \lim_{x\rightarrow 1}\frac{\abs{x-1}}{x-1}
\end{align*}
Não existe, vendo que os limites laterais diferem.
\subsection{Analise e rascunhe a função: \(f(x)=x^2\) para \(x \leq 1\), e \(f(x)=1\) para \(x > 1\)}
\begin{tikzpicture}
	\begin{axis}[axis lines=middle, samples=200]
        \addplot[domain=-2:1]{x^2};
        \addplot[domain=1:3]{1};
	\end{axis}
\end{tikzpicture}

A derivada só não existe no ponto \(x=1\), onde apresenta bico. A derivada não existe nesse ponto por conta dos limites laterais diferirem no ponto, visto que cada lado tem definições diferentes.

\part{Aula X Parte II}
\setcounter{section}{0}

\section{Assista a aula}

\section{}

\subsection{Grave um vídeo com um exemplo}
Calcule \(\frac{d^2y}{dx^2}\), sendo \(y = \cos{5x}\).
\begin{align*}
    (\cos{5x})'
    &= (5x)' (\cos{5x})' \\
    &= 5 \cdot (-\sin{5x})' \\
    &= -5 \sin{5x}
\end{align*}
\begin{align*}
    (-5 \sin{5x})'
    &= -5 (\sin{5x})' \\
    &= -5 \cdot (5x)' \cdot (\sin{5x})' \\
    &= -5 \cdot 5 \cdot \cos{5x} \\
    &= -25 \cos{5x}
\end{align*}


\part{Aula X+1: Regras de derivação}
\setcounter{section}{0}
\section{Assista os vídeos}
\subsection{Detalhe a derivação de \(f(x) = \sin{x}+x\cos{x}\)}
\begin{align*}
    (\sin{x}+x\cos{x})'
    &= (\sin{x})' + (x\cos{x})' \text{, [regra da soma] } \\
    &= \cos{x} + (x\cos{x})' \text{, [derivada de seno] } \\
    &= \cos{x} + (x)'\cdot\cos{x} + x \cdot (\cos{x})' \text{, [regra do produto] } \\
    &= \cos{x} + 1\cos{x} + x \cdot (-\sin{x}) \\
    &= 2\cos{x} - x\sin{x}
\end{align*}

\subsection{Detalhe a derivação de \(\cot{x}\), \(\tan{x^2}\) e \(\tan^2{x}\)}

Para facilitar a letra b e c, vamos determinar a derivada de \(\tan{x}\):

\begin{align*}
    (\tan{x})'
    &= \left( \frac{\sin{x}}{\cos{x}} \right)' \\
    &= \frac{(\sin{x})' \cdot \cos{x} - \sin{x} \cdot (\cos{x})'}{\cos^2{x}} \text{, [regra do quociente] } \\
    &= \frac{\cos{x}\cdot\cos{x} - \sin{x}(-\sin{x})}{\cos^2{x}} \\
    &= \frac{\cos^2{x} + \sin^2{x}}{\cos^2{x}} \\
    &= \frac{1}{\cos^2{x}} \\
    &= \sec^2{x}
\end{align*}

\subsubsection{\(\cot{x}\)}
\begin{align*}
    (\cot{x})'
    &= \left(\frac{\cos{x}}{\sin{x}}\right)' \\
    &= \frac{(\cos{x})'\cdot\sin{x}-\cos{x}\cdot(\sin{x})'}{\sin^2(x)} \text{, [regra do quociente] }\\
    &= \frac{(-\sin{x})\sin{x}-\cos{x}\cos{x}}{\sin^2{x}} \\
    &= \frac{-\sin^2{x}-\cos^2{x}}{\sin^2{x}} \\
    &= -\frac{1}{\sin^2{x}}
\end{align*}

\subsubsection{\(\tan{x^2}\)}
\begin{align*}
    (\tan{x^2})'
    &= \sec^2{x^2}\cdot (x^2)' \text{, [regra da cadeia]} \\
    &= \sec^2{x^2}\cdot 2x
\end{align*}

\subsubsection{\(\tan^2{x}\)}
\begin{align*}
    (\tan^2{x})'
    &= 2\tan{x}\cdot (\tan{x})' \text{, [regra da cadeia e da potência]} \\
    &= 2\tan{x}\cdot \sec^2{x}
\end{align*}

\end{document}
