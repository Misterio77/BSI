\documentclass[12pt]{article}
\usepackage{amsmath}
\usepackage{amssymb}
\usepackage{commath}
\usepackage{systeme}
\usepackage{booktabs}
\usepackage{indentfirst}
\usepackage{mathtools}
\usepackage{pgfplots}
\usepackage{tikz}
\usepackage[portuguese]{babel}
\DeclarePairedDelimiter{\ceil}{\lceil}{\rceil}
%Reduce part, section and subsection font sizes
\usepackage[small]{titlesec}
\titleformat{\part}[display]
  {\normalfont\large\bfseries}{\partname\ \thepart}{14pt}{\Large}

\tikzset{
  jumpdot/.style={mark=*,solid},
  excl/.append style={jumpdot,fill=white},
  incl/.append style={jumpdot,fill=black},
}
\renewcommand\thesubsection{\alph{subsection})}% chktex 9 chktex 10
\title{SMA0501 - Cálculo I \\ P1 - Parte 1}% chktex 8
\author{Gabriel Fontes - 10856803}% chktex 8

\begin{document}
\maketitle
Eu, Gabriel Silva Fontes, estou apresentando, a seguir, minhas próprias soluções para as atividades propostas na P1 - PARTE I. % chktex 8
Atesto que me dediquei as mesmas, seja individualmente, seja em colaboração com outro/as colegas.
Nos casos em que a discussão com colegas tenha resultado em soluções descritas de modo semelhante às de outro/as aluno/as, atesto que entendo a resolução apresentada no material individual que estou entregando.
\section{}
Gráfico \(y = \ceil{x}\):

\begin{tikzpicture}
	\begin{axis}[axis lines=middle]
	\addplot+[jump mark right,samples at={-5,-4,...,5}] {ceil(x)};
	\end{axis}
\end{tikzpicture}

\bigskip
\bigskip
\bigskip
\bigskip
\bigskip
\bigskip

Para uma função ser contínua num ponto, o limite precisa existir nesse ponto. Isto é, o limite pela direita e o limite pela esquerda devem ser iguais. Vamos tomar \(x=1\). Como é visível no gráfico:
\begin{align*}
	\lim_{x\rightarrow 1+} \ceil{x} &= \lim_{x\rightarrow 1-} \ceil{x} \\
	2 &= 1
\end{align*}
Contradição. \(f\) não é contínua em \(x=1\), logo não é contínua em todo \(p \in \mathbb{Z}\).

Caso você tome \(p\) como membro de um conjunto que inclui apenas um inteiro \(n\) e seu sucessor, esse \(p\), obrigatoriamente, faz parte de apenas uma das linhas horizontais do gráfico (mais especificamente, faz parte do intervalo delimitado por \(n\) e \(n+1\)). Nesse intervalo, o único valor que \(f(p)\) pode assumir é o próprio \(n\). Dessa maneira, temos uma função constante, que é contínua.

\section{}
Para que \(f\) seja contínua no ponto \(x=-2\), precisamos que \(f(-2)\) seja igual o limite da função nos arredores, isto é:
\[
	\ell = f(-2) = \lim_{x\rightarrow -2} \frac{x^3 + 4x^2 + 3x - 2}{x+2}
\] 
Pelas propriedades de limite, sabemos que o limite com \(x\) tendendo a \(-2\) de uma função \(f\) é igual ao de sua forma simplificada \(g\) (já que os arredores de \(x = -2\) são idênticos). Vamos simplificar \(f\) de forma a obter uma função \(g\) contínua em \(x=-2\).

Vamos tentar dividir o numerador pelo denominador. Para \(x \ne -2\):
\begin{align*}
	f(x) &= \frac{x^3 + 4x^2 + 3x - 2}{x+2} \\
	&= \frac{x^3 + 2x^2}{x+2} + \frac{2x^2 + 3x - 2}{x+2} \\
	&= x^2 + \frac{2x^2 + 3x - 2}{x+2} \\
	&= x^2 + \frac{2x^2 + 4x}{x+2} + \frac{-x-2}{x+2} \\
	&= x^2 + 2x + \frac{-x-2}{x+2} \\
	= g(x)
	&= x^2 + 2x - 1
\end{align*}
Agora podemos obter \(g(-2)\), já que \(g\) é contínua em \(-2\):
\begin{align*}
	g(-2) 
	&= {(-2)}^2 + 2(-2) -1 \\
	&= 4 - 4 - 1 \\
	&= -1
\end{align*}
Sendo assim:
\begin{align*}
	\ell = f(-2)
	&= \lim_{x\rightarrow -2} \frac{x^3 + 4x^2 + 3x - 2}{x+2} \\
	&= \lim_{x\rightarrow -2} x^2 + 2x - 1 \\
	&= {(-2)}^2 + 2(-2) -1 \\
	&= -1
\end{align*}
\section{}
\subsection{}
Vamos simplificar \(f\) para obter uma \(g\) idêntica nos arredores de \(x=1/5\), porém contínua no ponto.
Vamos começar tentando dividindo o numerador pelo denominador. Para \(x \ne 1/5\):
\begin{align*}
	f(x)
	&= \frac{10x^2 + 13x - 3}{5x-1} \\
	&= \frac{10x^2 - 2x}{5x - 1} + \frac{15x - 3}{5x-1} \\
	= g(x)
	&= 2x + 3
\end{align*}
Como sabemos, caso \(f\) e \(g\) sejam iguais nos arredores de \(p\), e com \(g\) contínua nesse mesmo \(p\):
\[
	\lim_{x\rightarrow p} f(x) = \lim_{x\rightarrow p} g(x) = g(p)
\]
Assim:
\begin{align*}
	\lim_{x\rightarrow 1/5} \frac{10x^2 + 13x - 3}{5x-1} 
	&= \lim_{x\rightarrow 1/5} 2x + 3\\
	&= 2\cdot \frac{1}{5} + 3 \\
	&= \frac{17}{5}
\end{align*}
\subsection{}
Usando a mesma justificativa da letra a), vamos simplificar a função. Para \(x \ne 3\):
\begin{align*}
	f(x)
	&= \frac{x^4-3^4}{x-3} \\
	&= \frac{{(x^2)}^2-9^2}{x-3} \\
	&= \frac{(x^2+9)\cdot(x^2-9)}{x-3} \\
	&= \frac{(x^2+9)\cdot(x+3)\cdot(x-3)}{x-3} \\
	= g(x)
	&= (x^2+9)\cdot(x+3)
\end{align*}
Agora:
\begin{align*}
	\lim_{x\rightarrow 3} \frac{x^4-3^4}{x-3}
	&= \lim_{x\rightarrow 3} (x^2+9)\cdot(x+3) \\
	&= (3^2+9)\cdot(3+3) \\
	&= 18\cdot6 \\
	&= 108
\end{align*}
\subsection{}
Mais uma vez, com a mesma justificativa, vamos simplificar esta função. Para \(x \ne 1\):
\begin{align*}
	f(x)
	&= \frac{\sqrt{x}-1}{\sqrt{3x-1}-\sqrt{2}} \\
	&= \frac{\sqrt{x}-1}{\sqrt{3x-1}-\sqrt{2}} \cdot \frac{\sqrt{3x-1}+\sqrt{2}}{\sqrt{3x-1}+\sqrt{2}} \\
	&= \frac{(\sqrt{x}-1)\cdot(\sqrt{3x-1}+\sqrt{2})}{3x-1-2} \\
	&= \frac{(\sqrt{x}-1)\cdot(\sqrt{3x-1}+\sqrt{2})}{3\cdot(x-1)} \\
	&= \frac{(\sqrt{x}-1)\cdot(\sqrt{3x-1}+\sqrt{2})}{3\cdot((\sqrt{x}-1)\cdot(\sqrt{x}+1))} \\
	= g(x)
	&= \frac{\sqrt{3x-1}+\sqrt{2}}{3\cdot(\sqrt{x}+1)}
\end{align*}
Assim, temos:
\begin{align*}
	\lim_{x\rightarrow 1} \frac{\sqrt{x}-1}{\sqrt{3x-1}-\sqrt{2}}
	&= \lim_{x\rightarrow 1} \frac{\sqrt{3x-1}+\sqrt{2}}{3\cdot(\sqrt{x}+1)} \\
	&= \frac{\sqrt{3-1}+\sqrt{2}}{3\cdot(1+1)} \\
	&= \frac{\sqrt{2}+\sqrt{2}}{3\cdot2} \\
	&= \frac{2\sqrt{2}}{2\cdot3} \\
	&= \frac{\sqrt{2}}{3}
\end{align*}
\subsection{}
Com o mesmo raciocínio das anteriores, vamos simplificar. Para \(x \ne 2\):
\begin{align*}
	f(x)
	&= \frac{\sqrt[3]{x}-\sqrt[3]{2}}{x-2} \\
	&= \frac{\sqrt[3]{x}-\sqrt[3]{2}}{x-2}\cdot\frac{2^{2/3}+\sqrt[3]{2}\sqrt[3]{x}+x^{2/3}}{2^{2/3}+\sqrt[3]{2}\sqrt[3]{x}+x^{2/3}} \\
	= g(x)
	&= \frac{1}{2^{2/3}+\sqrt[3]{2}\sqrt[3]{x}+x^{2/3}}
\end{align*}
Agora:
\begin{align*}
	\lim_{x\rightarrow 2} \frac{\sqrt[3]{x}-\sqrt[3]{2}}{x-2}
	&= \lim_{x\rightarrow 2} \frac{1}{2^{2/3}+\sqrt[3]{2}\sqrt[3]{x}+x^{2/3}} \\
	&= \frac{1}{2^{2/3}+\sqrt[3]{2}\sqrt[3]{2}+2^{2/3}} \\
	&= 3\cdot2^{2/3} \\
	&= \frac{\sqrt[3]{2}}{6}
\end{align*}
\section{Desafio}
TODO
\end{document} 
